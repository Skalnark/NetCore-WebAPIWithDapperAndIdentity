\documentclass{article}
\usepackage[utf8]{inputenc}
\usepackage{minted}

\title{HOW TO BUILD A WEB API WITH .NET CORE AND DAPPER}
\author{skalnark}
\date{October 2020}

\begin{document}

\maketitle

\section{Autenticação}

\begin{enumerate}
    \item Criar projeto vazio
    
    \item Criar projeto de API
    
    \item Criar o databasecontext\\
        Esse item deriva da classe IdentityDbContext
        
    \item Adicionar isso no startup.cs:
    
    \begin{minted}{csharp}
    
services.AddIdentity<IdentityUser, IdentityRole>()
    .AddEntityFrameworkStores<DatabaseContext>()
    .AddDefaultTokenProvider();
    \end{minted}

    \item Adicionar os midlewares de autenticação e autorização no Configure:
    
    \begin{minted}{csharp}
app.UseAuthentication();
app.UseAuthorization();
    \end{minted}

    \item Adicionar construtor padrão vazio e o seguinte construtor no DatabaseContext:
    
    \begin{minted}{csharp}
public DatabaseContext(DbContextOptions<DatabaseContext> options)
        : base(options)
    {  }
    \end{minted}
    \item Adicionar o seguinte no ConfigureServices da Startup.cs:
    
    \begin{minted}{csharp}
services.AddDbContext<DatabaseContext>(options =>
options.UseNpgsql(Configuration.GetConnectionString("DefaultConnection")));
    \end{minted}
    
    \item Adicionar a seguinte configuração no appsettings.json:
    
    \begin{minted}{json}
  "ConnectionStrings": {
    "DefaultConnectioon": "Host=127.0.0.1;Port=5432;Pooling=true;
    Database=NOME DO BANCO DE DADOS;
    Username=postgres;Password=password"
  },
    \end{minted}
    
    \item Criar UserDTO com email, senha e confirma senha
    
    \item Criar Controller de autorização com a seguintes variáveis:
    
    \begin{minted}{csharp}
private readonly UserManager<IdentityUser> userManager;
private readonly SignInManager<IdentityUser> signInManager;
private readonly IConfiguration configuration;
    \end{minted}
    
    \item Criar método Post RegisterUser com o seguinte corpo:
    
    \begin{minted}{csharp}
[HttpPost("register")]
public async Task<ActionResult> RegisterUser([FromBody]UserDTO model)
{
    var user = new IdentityUser
    {
        UserName = model.Email,
        Email = model.Email,
        EmailConfirmed = true
    };

    var result = await userManager.CreateAsync(user, model.Password);

    if (!result.Succeeded)
    {
        return BadRequest(result.Errors);
    }

    await signInManager.SignInAsync(user, false);

    return Ok(TokenGenerator(model));
}
    \end{minted}
    
    \item Add Login:
    
    \begin{minted}{csharp}
[HttpPost("login")]
public async Task<ActionResult> Login([FromBody] UserDTO model)
{
    var result = await signInManager.PasswordSignInAsync(
    userName: model.Email, password: model.Password, isPersistent: false, false);

    if (result.Succeeded)
    {
        return Ok(TokenGenerator(model));
    }

    ModelState.AddModelError(string.Empty, "incorrect email or password");
    return BadRequest(ModelState);
}
    \end{minted}
    
    \item Instalar o \textbf{System.IdentityModel.Tokens.Jwt}
    
    \item Adicionar no appsettings.json:
    
    \begin{minted}{json}
    "Jwt": {
      "Key": "CHAVE SECRETA PODE GERAR COM date | openssl sha3-256"
    },
    "TokenConfiguration": {
      "Audiece": "Nome_Audience",
      "Issuer": "Nome_Issuer",
      "ExpireHours": 2
    }
    \end{minted}
    
    \item Criar DTO UserToken com o seguinte código:
    \begin{minted}{csharp}
    public class UserToken
    {
        public bool Authenticated { get; set; }
        public DateTime Expiration { get; set; }
        public string Token { get; set; }
        public string Message { get; set; }
    }
    \end{minted}
    
    \item Criar o método TokenGenerator com o seguinte código:
    
    \begin{minted}{csharp}
    private UserToken TokenGenerator(UserDTO model)
        {
            var claims = new[]
            {
                new Claim(JwtRegisteredClaimNames.UniqueName, model.Email)
            };

            var key = new SymmetricSecurityKey(
                Encoding.UTF8.GetBytes(configuration["Jwt:Key"]));

            var credentials = new SigningCredentials(key, SecurityAlgorithms.HmacSha256);

            var expirationTime = configuration["TokenConfiguration:ExpireHours"];
            var expiration = DateTime.UtcNow.AddHours(double.Parse(expirationTime));

            JwtSecurityToken token = new JwtSecurityToken(
                issuer: configuration["TokenConfiguration:Issuer"],
                audience: configuration["TokenConfiguration:Audience"],
                claims: claims,
                expires: expiration,
                signingCredentials: credentials
                );

            return new UserToken()
            {
                Authenticated = true,
                Token = new JwtSecurityTokenHandler().WriteToken(token),
                Expiration = expiration,
                Message = "New JWT token"
            };
        }
    \end{minted}
    
    \item Instalar o \textbf{Microsoft.AspNetCore.Authentication.JwtBearer}
    
    \item importar o \textbf{Microsoft.IdentityModel.Tokens}
    
    \item Inserir depois do AddIdentity:
    
    \begin{minted}{csharp}
services.AddAuthentication(JwtBearerDefaults.AuthenticationScheme)
            .AddJwtBearer(options =>
            options.TokenValidationParameters = new TokenValidationParameters
            {
                ValidateIssuer = true,
                ValidateAudience = true,
                ValidateLifetime = true,
                ValidAudience = Configuration["TokenConfiguration:Audience"],
                ValidIssuer = Configuration["TokenConfiguration:Issuer"],
                ValidateIssuerSigningKey = true,
                IssuerSigningKey = new SymmetricSecurityKey(
                    Encoding.UTF8.GetBytes(Configuration["Jwt:Key"]))
                });
    \end{minted}
    
    \item Criar Migration \textbf{Add-Migration Identity}
    
    \item Atualizar banco de dados \textbf{Update-Database}
    
    \item Para autenticar operações em um controller, é necessário adicionar o seguinte: \\
    \textbf{
        [Authorize(AuthenticationSchemes = "Bearer")]}
\end{enumerate}

\section{Aplicação}

\begin{enumerate}

    \item Gerar o banco de dados
    
    \item Criar as entidades

    \item Adicionar DbSet<Entidade> no DatabaseContext
    
    \item Criar IRepository com o seguinte código:
    
    \begin{minted}{csharp}
public interface IRepository<T>
    where T : class
{
    Task<IEnumerable<T>> Get();
    Task<T> Get(int id);

    Task<int> Add(T entity);
    Task<int> Update(T entity);
    void Delete(int id);
}
    \end{minted}
    
    \item Criar uma implementação do repository para cada entidade contendo as seguintes dependencias:
    
    \begin{minted}{csharp}

private readonly IConfiguration _configuration;
    \end{minted}
    
    \item Instalar o Dapper
    
    \item Exemplo de Add:
    \begin{minted}{csharp}
public async Task<int> Add(Product entity)
    {
        string sql = "INSERT INTO Products (ProductName, Price, CategoryId) VALUES (@ProductName, @Price, @CategoryId) RETURNING ProductId";

        using var connection = new Npgsql.NpgsqlConnection(_configuration.GetConnectionString("DefaultConnection"));
        connection.Open();

        return await connection.ExecuteAsync(sql, entity);
    }
    \end{minted}
    
    \item Exemplo de Delete:
    \begin{minted}{csharp}
public async void Delete(int id)
    {
        var sql = "DELETE FROM Products WHERE ProductId = @ProductId";

        using var connection = new Npgsql.NpgsqlConnection(_configuration.GetConnectionString("DefaultConnection"));
        await connection.OpenAsync();
        var result = connection.QueryAsync<Product>(sql, new { ProductId = id });
    }
    \end{minted}
    
    \item Exemplo de Get:
    \begin{minted}{csharp}
public async Task<IEnumerable<Product>> Get()
{
    string sql = "SELECT * FROM Products";
    using var connection = new NpgsqlConnection(_configuration.GetConnectionString("DefaultConnection"));
    await connection.OpenAsync();

    return await connection.QueryAsync<Product>(sql);
}

public async Task<Product> Get(int id)
{
    string sql = "SELECT * FROM Products WHERE ProductId = @ProductId";

    using var connection = new NpgsqlConnection(_configuration.GetConnectionString("DefaultConnection"));
    await connection.OpenAsync();

    var result = connection.QueryAsync<Product>(sql, new { ProductId = id }).Result.FirstOrDefault();

    return result;
}
    \end{minted}

    \item Exemplo de Update:
    \begin{minted}{csharp}
public async Task<int> Update(Product entity)
{
    var sql = @"UPDATE Product SET " +
        "ProductName = @ProductName, " +
        "Price = @Price, " +
        "CategoryId = @CategoryId, " +
        "Phone = @Phone " +
        "WHERE ProductId = @ProductId";

    using var connection = new NpgsqlConnection(_configuration.GetConnectionString("DefaultConnection"));
    await connection.OpenAsync();

    return await connection.ExecuteAsync(sql, entity);
}
    \end{minted}
    
    \item Implementar o IUnitOfWork e UnitOfWork com o seguinte código:
    
    \begin{minted}{csharp}
public interface IUnitOfWork
{
    IRepositoryProduct Product { get; }
    IRepositoryCategory Categories { get; }
    Task Commit();
    void Dispose();
}

public class UnitOfWork : IUnitOfWork
{
    private readonly IConfiguration _configuration;
    private readonly DatabaseContext _context;
    private RepositoryPerson _people;

    public UnitOfWork(IConfiguration configuration,DatabaseContext context)
    {
        _context = context;
        _configuration = configuration;
    }

    public IRepositoryPerson People
    => _people ??= new RepositoryPerson(_configuration, _context);
        
    public async Task Commit() => await _context.SaveChangesAsync();

    public void Dispose() => _context.Dispose();
}

    \end{minted}
    
    \item Adicionar no Startup:
    \begin{minted}{csharp}
services.AddTransient<IRepositoryProduct, RepositoryProduct>();
services.AddTransient<IRepositoryCategory, RepositoryCategory>();
services.AddTransient<IUnitOfWork, UnitOfWork>();
    \end{minted}
\end{enumerate}
\end{document}
